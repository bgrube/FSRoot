\documentclass[11pt]{article}
%\usepackage{graphicx}
\usepackage{amssymb}
%\usepackage[]{epsfig}
%\usepackage{parskip}

\textheight 8.0in
\topmargin 0.0in
\textwidth 6.5in
\oddsidemargin 0.0in
\evensidemargin 0.0in

\newcommand{\FSR}{{\tt FSRoot}}
\newcommand{\ROOT}{{\tt ROOT}}
\newcommand{\git}{{\tt git}}

\begin{document}

% works down to v5.34.09
% checks for 6.18
% less than 6.18 can't use:
%   FSSystem::makeAbsolutePathName(TString path)
%   RDataFrame methods
%   FSHistogram::getTH1FFormula(TF1* function, TString bounds, int numRandomTrials)

\title{Notes on the \FSR\ Package}
\author{Ryan Mitchell}
\date{\today~~~(v3.0+)}
\maketitle

\abstract{
\FSR\ is a set of utilities to help manipulate information about different Final States~(FS) produced in particle physics experiments.  The utilities are built around the CERN \ROOT\ framework.  This document provides an introduction to \FSR.}

\tableofcontents

\parindent 0pt
\parskip 10pt


\newpage

%\section{Overview of \FSR}

%This is an overview of \FSR.

\section{Installation and Initial Setup}

The \FSR\ utilities are built around the \ROOT\ framework, so a working version of \ROOT\ is a prerequisite.  Notes on \ROOT\ versions: \\
-- {\tt v6.18} and after: \FSR\ should work. \\
-- {\tt v5.34.09} up to {\tt v6.18}: \FSR\ should work, but without some functionality (e.g. {\tt RDataFrame}). \\
-- {\tt v5.34.08} and before: \FSR\ may or may not work.

The \FSR\ source code is on {\tt GitHub}: \\ {\tt https://github.com/remitche66/FSRoot}

\begin{enumerate}

\item To download the current working version, use {\tt git clone}:
\begin{verbatim}
    > git clone https://github.com/remitche66/FSRoot.git FSRoot
\end{verbatim}
Optionally, to switch to a specific tag ({\tt v3.0} for example):
\begin{verbatim}
    > git checkout tags/v3.0
\end{verbatim}
Alternatively, download and unpack ({\tt tar -xzf}) a released version from here:
\begin{verbatim}
    https://github.com/remitche66/FSRoot/releases
\end{verbatim}


\item Set the location of \FSR\ in your login shell script (e.g. {\tt .cshrc}):
\begin{verbatim}
    setenv FSROOT [path]/FSRoot
\end{verbatim}

\item To build a static library (which will appear as {\tt \$FSROOT/lib/libFSRoot.a}) use:
\begin{verbatim}
    > cd $FSROOT
    > make
\end{verbatim}

\item To use \FSR\ interactively within a \ROOT\ session, add these lines to your {\tt .rootrc} file (usually found in your home directory):
\begin{verbatim}
    Unix.*.Root.DynamicPath: .:$(FSROOT):$(ROOTSYS)/lib:
    Unix.*.Root.MacroPath:   .:$(FSROOT):
    Rint.Logon:    $(FSROOT)/rootlogon.FSROOT.C
    Rint.Logoff:   $(FSROOT)/rootlogoff.FSROOT.C
\end{verbatim}
The last two lines load and unload \FSR\ when you open and close \ROOT.
Alternatively, you could load \FSR\ manually from \ROOT:
\begin{verbatim}
    root> .x $FSROOT/rootlogon.FSROOT.C
\end{verbatim}
When \FSR\ is loaded and compiled, you should see a message saying ``Loading the FSRoot Macros'' along with output from the compilation.  [Note that the {\tt rootlogon.FSROOT.C} file also sets up default styles, which are not essential.  If these conflict with styles you have defined elsewhere, you can tweak or remove these.]

\item It may sometimes be necessary to add the \FSR\ directory to library path variables:
\begin{verbatim}
    setenv DYLD_LIBRARY_PATH $DYLD_LIBRARY_PATH\:$FSROOT
    setenv   LD_LIBRARY_PATH   $LD_LIBRARY_PATH\:$FSROOT
\end{verbatim}

\end{enumerate}


\section{Basic Operations:  the {\tt FSBasic} directory}

\subsection{Basic Conventions: the {\tt TTree} format}
\label{sec:conventions}

Some \FSR\ operations on \ROOT\ {\tt TTree} variables assume a particular format for the {\tt TTree}.  \FSR\ generally expects a {\tt TTree} with branches holding numbers (usually {\tt double}).

Four-vectors are assumed to have the form:
\begin{verbatim}
    [AB]EnP[CD], [AB]PxP[CD], [AB]PyP[CD], [AB]PzP[CD]
\end{verbatim}
where {\tt [CD]} is a particle label (often {\tt 1}, {\tt 2}, {\tt 3}, {\tt 2a}, etc., but it could also be {\tt CM} or {\tt BEAM} or anything else) and {\tt [AB]} labels the type of four-vector (for example, it could be {\tt R} for raw or {\tt K} for kinematically fit or {\tt MC} for Monte Carlo, etc.).  The \FSR\ code has no requirements on {\tt [AB]} -- it could be anything up to two characters long~(or nothing).  For {\tt [CD]}, the final state utilities described in Sec.~\ref{sec:modes} use the numbering convention described in Sec.~\ref{sec:modeconv}.  
Otherwise, {\tt [CD]} could be anything and any length~(but not empty).

For convenience, variables associated with a given particle~(like the $\chi^2$ of a track) should use the same particle labels {\tt [CD]} as above.  For example, for a pion with four-momentum {\tt EnP5}, {\tt PxP5}, {\tt PyP5}, {\tt PzP5}, the corresponding track $\chi^2$ could be {\tt TkChi2P5}.

To use MC tagging utilities, \FSR\ uses tree variables named {\tt MCDecayCode1}, {\tt MCDecayCode2}, and {\tt MCExtras} (see Sec.~\ref{sec:mccomponents}).


\subsection{Basic Histogram Utilities: the {\tt FSHistogram} class}
\label{sec:hist}

Basic histogram functions are provided by the {\tt FSHistogram} class in the {\tt FSBasic} directory.  Like most other functions within \FSR, the functions within the {\tt FSHistogram} class are static member functions, so there is never a need to deal with instances of {\tt FSHistogram}.

The basic functions are {\tt FSHistogram::getTH1F} and {\tt FSHistogram::getTH2F}.  Here are examples that can be run from either the \ROOT\ command line or from a macro~(see {\tt Examples/Intro/intro.C}).  The first draws a 1d histogram and the second draws a 2d histogram.  The third argument is the variable to plot; the fourth holds the number of bins and bounds.
\begin{verbatim}
    > FSHistogram::getTH1F(fileName,treeName,"Chi2DOF","(60,0.0,6.0)","")->Draw();
    > FSHistogram::getTH2F(fileName,treeName,"Chi2DOF:Event",
         "(100,0.0,100.0,100,0.0,6.0)","")->Draw("colz");
\end{verbatim}
Cuts can be added in the fifth argument:
\begin{verbatim}
    > FSHistogram::getTH1F(fileName,treeName,"Chi2DOF",
        "(60,0.0,6.0)","Chi2DOF<2.0")->Draw();
\end{verbatim}
The variable and cut arguments can contain shortcuts.  For example, {\tt MASS(1,2)} is expanded into the total invariant mass of particles {\tt 1} and {\tt 2}, where {\tt 1} and {\tt 2} are not necessarily numbers~(they are the {\tt [CD]} in Sec.~\ref{sec:conventions}).  Characters in front of {\tt MASS}~(for example) are prepended to the variable names (the {\tt [AB]} in Sec.~\ref{sec:conventions}).  This is all done in the function {\tt FSTree::expandVariable}, which can also be run independently for testing.  
A list of all macros can be seen using {\tt FSTree::showDefinedMacros}~(for example, {\tt RECOILMASS}, {\tt MOMENTUM}, {\tt COSINE}, etc.).  An example for {\tt FSHistogram::getTH1F}:
\begin{verbatim}
    > FSHistogram::getTH1F(fileName,treeName,"MASS(2,4,5)",
        "(100,3.0,3.2)","Chi2DOF<2.0&&abs(MASS(2,4)-1.0)<0.2")->Draw();
\end{verbatim}
Examples for {\tt FSTree::expandVariable}:
\begin{verbatim}
    > cout << FSTree::expandVariable("MASS(1,2)+MASS(2,3)") << endl;
      // ==> "(sqrt(pow(((EnP1+EnP2)),2)-pow(((PxP1+PxP2)),2)
                   -pow(((PyP1+PyP2)),2)-pow(((PzP1+PzP2)),2)))
             +(sqrt(pow(((EnP2+EnP3)),2)-pow(((PxP2+PxP3)),2)
                   -pow(((PyP2+PyP3)),2)-pow(((PzP2+PzP3)),2)))"

    > cout << FSTree::expandVariable("ABMASS(1,C)") << endl;
      // ==> "(sqrt(pow(((ABEnP1+ABEnPC)),2)-pow(((ABPxP1+ABPxPC)),2)
                   -pow(((ABPyP1+ABPyPC)),2)-pow(((ABPzP1+ABPzPC)),2)))"

    > cout << FSTree::expandVariable("RECOILMASS(CM;D,2)") << endl;
      // ==> "(sqrt(pow(((EnPCM)-(EnP2+EnPD)),2)-pow(((PxPCM)-(PxP2+PxPD)),2)
                   -pow(((PyPCM)-(PyP2+PyPD)),2)-pow(((PzPCM)-(PzP2+PzPD)),2)))"
\end{verbatim}
Histograms are automatically cached so they are made only once.  To save histograms at the end of a session, use the function:
\begin{verbatim}
    > FSHistogram::dumpHistogramCache(); 
\end{verbatim}
To read in the cache at the beginning of a session:
\begin{verbatim}
    > FSHistogram::readHistogramCache(); 
\end{verbatim}
To clear a cache from memory during a session:
\begin{verbatim}
    > FSHistogram::clearHistogramCache(); 
\end{verbatim}
To see the contents of the cache:
\begin{verbatim}
    > FSHistogram::showHistogramCache(); 
\end{verbatim}

\subsection{Basic Cut Utilities: the {\tt FSCut} class}
\label{sec:cut}

Additional shortcuts for making plots are available through the {\tt FSCut} class in the {\tt FSBasic} directory.  The following example~(not using {\tt FSCut}) defines two cuts and then uses them to make a plot, as above:
\begin{verbatim}
    > TString cutCHI2("Chi2DOF<2.0");
    > TString cutMASS("abs(MASS(2,4)-1.0)<0.2");
    > FSHistogram::getTH1F(fileName,treeName,"MASS(2,4,5)",
        "(100,3.0,3.2)",cutCHI2+"&&"+cutMASS)->Draw();
\end{verbatim}
As a shortcut to do the exact same thing, give the cuts names and then implement them using the keyword {\tt CUT()}:
\begin{verbatim}
    > FSCut::defineCut("chi2",cutCHI2);
    > FSCut::defineCut("mass",cutMASS);
    > FSHistogram::getTH1F(fileName,treeName,"MASS(2,4,5)",
        "(100,3.0,3.2)","CUT(chi2,mass)")->Draw();
\end{verbatim}
The {\tt FSCut} class can also be used to define sidebands, which can then be implemented using the keyword {\tt CUTSB()}:
\begin{verbatim}
    > TString cutCHI2SB("Chi2DOF>3.0&&Chi2DOF<7.0");
    >   // the relative size of signal to sideband is 0.5
    > FSCut::defineCut("chi2",cutCHI2,cutCHI2SB,0.5);
    > FSHistogram::getTH1F(fileName,treeName,"MASS(2,4,5)",
        "(100,3.0,3.2)","CUT(mass)&&CUTSB(chi2)")->Draw();
\end{verbatim}
If multiple sidebands are used simultaneously, then all combinations of sideband regions are considered and the resulting histogram is a sum of sideband regions with weights determined by the weights for individual regions:
\begin{verbatim}
    > TString cutMASSSB("abs(MASS(2,4)-1.8)<0.6");
    >   // the relative size of signal to sideband is 1.0/3.0
    > FSCut::defineCut("mass",cutMASS,cutMASSSB,1.0/3.0);
    > FSHistogram::getTH1F(fileName,treeName,"MASS(2,4,5)",
        "(100,3.0,3.2)","CUTSB(chi2,mass)")->Draw();
\end{verbatim}


\subsection{Basic Tree Utilities: the {\tt FSTree} class}
\label{sec:tree}

The {\tt FSTree} class is also located in the {\tt FSBasic} directory and provides basic utilities to operate on trees.  
Besides the static {\tt FSTree::expandVariable} member function mentioned in Sec.~\ref{sec:hist}, the most useful function is for skimming trees.  For example:
\begin{verbatim}
    > FSTree::skimTree(inputFileName, inputTreeName, outputFileName,
          "Chi2DOF<5.0"); 
\end{verbatim}
This will take the tree named {\tt inputTreeName} from the file {\tt inputFileName}, loop over all events and select only those that pass the cut on {\tt Chi2DOF}, then output the selected events to the file named {\tt outputFileName} in a tree with the same name as the input tree.  The shortcuts mentioned in Sec~\ref{sec:hist} can also be used here, for example:
\begin{verbatim}
    > FSTree::skimTree(inputFileName, inputTreeName, outputFileName,
          "abs(MASS(1,2)-0.5)<0.1"); 
\end{verbatim}


\section{Final State Operations:  the {\tt FSMode} directory}
\label{sec:modes}

\subsection{Mode Numbering and Conventions}
\label{sec:modeconv}

A ``final state''~(also called ``mode'') is made from a combination of: $\Lambda (\to p \pi^-)$, $\bar{\Lambda} (\to \bar{p} \pi^+)$, $e^+$, $e^-$, $\mu^+$, $\mu^-$, $p$, $\bar{p}$, $\eta (\to \gamma\gamma)$, $\gamma$, $K^+$, $K^-$, $K^0_S (\to \pi^+\pi^-)$, $\pi^+$, $\pi^-$, $\pi^0 (\to \gamma\gamma)$.

As strings, these final state particles are:
$\Lambda (\to p \pi^-) \equiv $~{\tt Lambda}, 
$\bar{\Lambda} (\to \bar{p} \pi^+) \equiv $~{\tt ALambda}, 
$e^+ \equiv $~{\tt e+}, 
$e^- \equiv $~{\tt e-}, 
$\mu^+ \equiv $~{\tt mu+},  
$\mu^- \equiv $~{\tt mu-}, 
$p \equiv $~{\tt p+}, 
$\bar{p} \equiv $~{\tt p-}, 
$\eta (\to \gamma\gamma) \equiv $~{\tt eta}, 
$\gamma \equiv $~{\tt gamma}, 
$K^+ \equiv $~{\tt K+}, 
$K^- \equiv $~{\tt K-}, 
$K^0_S (\to \pi^+\pi^-) \equiv $~{\tt Ks}, 
$\pi^+ \equiv $~{\tt pi+}, 
$\pi^- \equiv $~{\tt pi-}, 
$\pi^0 (\to \gamma\gamma) \equiv $~{\tt pi0}.

In a {\tt TTree}, the final state particles should be listed in the order they are given above.  The numbering for the {\tt [CD]} (Sec.~\ref{sec:conventions}) starts at {\tt 1}.  For final state particles that decay~($\Lambda (\to p \pi^-) \equiv $~{\tt Lambda}, 
$\bar{\Lambda} (\to \bar{p} \pi^+) \equiv $~{\tt ALambda}, $\eta (\to \gamma\gamma) \equiv $~{\tt eta}, $K^0_S (\to \pi^+\pi^-) \equiv $~{\tt Ks}, $\pi^0 (\to \gamma\gamma) \equiv $~{\tt pi0}), the four-momenta of the decay particles are listed using {\tt a} and {\tt b} in the same order as above.  No assumptions about ordering are made for identical particles.

For example, for the final state $\gamma K^+ K^0_S \pi^+ \pi^- \pi^- \pi^0$, the four-momenta are:
\begin{verbatim}
    EnP1  PxP1  PyP1  PzP1     (for the gamma)
    EnP2  PxP2  PyP2  PzP2     (for the K+)
    EnP3  PxP3  PyP3  PzP3     (for the Ks)
    EnP3a PxP3a PyP3a PzP3a    (for the pi+ from Ks)
    EnP3b PxP3b PyP3b PzP3b    (for the pi- from Ks)
    EnP4  PxP4  PyP4  PzP4     (for the pi+)
    EnP5  PxP5  PyP5  PzP5     (for one pi-)
    EnP6  PxP6  PyP6  PzP6     (for the other pi-)
    EnP7  PxP7  PyP7  PzP7     (for the pi0)
    EnP7a PxP7a PyP7a PzP7a    (for one gamma from pi0)
    EnP7b PxP7b PyP7b PzP7b    (for the other gamma from pi0)
\end{verbatim}





Every final state can be specified in three different ways:

(1) {\tt pair<int,int> modeCode}: a pair of two integers ({\tt modeCode1}, {\tt modeCode2}) that count the number of particles in a decay mode:
\begin{verbatim}
    modeCode1 = abcdefg
      a = # gamma      d = # Ks         g = # pi0
      b = # K+         e = # pi+
      c = # K-         f = # pi-
    modeCode2 = abcdefghi
      a = # Lambda     d = # e-         g = # p+
      b = # ALambda    e = # mu+        h = # p-
      c = # e+         f = # mu-        i = # eta
\end{verbatim}
(2) {\tt TString modeString}: a string version of {\tt modeCode1} and {\tt modeCode2} in the format ``{\tt modeCode2\_modeCode1}''.  It can contain a prefix (for example, {\tt FS} or {\tt EXC} or {\tt INC} or anything longer) that isn't used here, but can help with organization elsewhere.  

(3) {\tt TString modeDescription}: a string with a list of space-separated particle names~(for example, ``{\tt K+ K- pi+ pi+ pi- pi-}'').  The final state particles can appear in any order. 

For example, the final state $\gamma K^+ K^0_S \pi^+ \pi^- \pi^- \pi^0$ has {\tt modeCode1 = 1101121}, {\tt modeCode2 = 0}, {\tt modeString = "0\_1101121"}, and {\tt modeDescription = "gamma K+ Ks pi+ pi- pi- pi0"}. 

\subsection{Mode Information: the {\tt FSModeInfo} class}
\label{sec:modeinfo}

Information about an individual final state is carried by the {\tt FSModeInfo} class.  Here are examples of a few of its basic member functions:
\begin{verbatim}
    > FSModeInfo mi("K+ K- K+ K- pi+ pi- pi0 eta");
    > mi.modeString();   // ==> "1_220111"
    > mi.modeCode1();    // ==> 220111
    > mi.modeCode2();    // ==> 1
    > mi.modeString("this is the code: (MODECODE1,MODECODE2)");
       // ==> "this is the code: (220111,1)" 
    > mi.modeString("MODESTRING corresponds to MODEDESCRIPTION");
       // ==> "1_220111 corresponds to  K+  K+  K-  K-  pi+  pi-  pi0  eta "
    > mi.modeString("The K- mesons have indices LIST([K-]).");
       // ==> "The K- mesons have indices 4,5."
\end{verbatim}

The {\tt FSModeInfo} class also handles particle combinatorics within a given final state through the {\tt modeCombinatorics} member function.  This is done using place holders like {\tt [pi+]}, {\tt [pi-]}, {\tt [K+]}, {\tt [K+3]}, {\tt [tk+]}, {\tt [pi0]}, etc., which are replaced by particle indices.  While the {\tt modeCombinatorics} function is rarely used explicitly by the user, it can be useful for cross-checking the behavior of the combinatorics. For example:
\begin{verbatim}
    > mi.modeCombinatorics("K+[K+], K-[K-], K+(again)[K+], K+(other one)[K+2]",true);
      // ==>  *******************************
      // ==>  *** MODE COMBINATORICS TEST ***
      // ==>  *******************************
      // ==>        mode =  K+  K+  K-  K-  pi+  pi-  pi0  eta 
      // ==>       input = K+[K+], K-[K-], K+(again)[K+], K+(other one)[K+2]
      // ==>    combinations:
      // ==>           (1) K+2,K-4,K+(again)2,K+(otherone)3
      // ==>           (2) K+2,K-5,K+(again)2,K+(otherone)3
      // ==>           (3) K+3,K-4,K+(again)3,K+(otherone)2
      // ==>           (4) K+3,K-5,K+(again)3,K+(otherone)2
      // ==>  *******************************
\end{verbatim}

The {\tt modeCuts} member function uses the results of {\tt modeCombinatorics} to combine combinatorics into a single string using the keywords {\tt AND}, {\tt OR}, {\tt MAX}, {\tt MIN}, and {\tt LIST}.  It is also called by the {\tt modeString} member function.  Examples:
\begin{verbatim}
    > cout << mi.modeCuts("OR((ABC[K+]+DEF[K-])>0)") << endl;
      // ==>  "(((ABC2+DEF4)>0)||((ABC2+DEF5)>0)||((ABC3+DEF4)>0)||((ABC3+DEF5)>0))"
    > cout << mi.modeCuts("AND((ABC[K+]+DEF[K-])>0)") << endl;
      // ==>  "(((ABC2+DEF4)>0)&&((ABC2+DEF5)>0)&&((ABC3+DEF4)>0)&&((ABC3+DEF5)>0))"
    > cout << mi.modeCuts("MAX(ABC[K+])") << endl;
      // ==>  "(((ABC[K+])>=(ABC2))&&((ABC[K+])>=(ABC3)))"
    > cout << mi.modeCuts("MIN(ABC[K+])") << endl;
      // ==>  "(((ABC[K+])<=(ABC2))&&((ABC[K+])<=(ABC3)))"
    > cout << mi.modeCuts("LIST(ABC[K+])") << endl;
      // ==>  "ABC2,ABC3"
\end{verbatim}

Note that most of the functionality of the {\tt FSModeInfo} class is rarely used explicitly.  It is more often combined with other functions and used in higher-level classes~(like {\tt FSModeHistogram}), often producing large strings~(used only internally), like:
\begin{verbatim}
    > FSTree::expandVariable(mi.modeCuts("AND(MASS2([K+],[K-])>MASS2([pi+],[K-]))"))
      // ==>  "(((pow(((EnP2+EnP4)),2)-pow(((PxP2+PxP4)),2)
                                      -pow(((PyP2+PyP4)),2)-pow(((PzP2+PzP4)),2))>
                 (pow(((EnP4+EnP6)),2)-pow(((PxP4+PxP6)),2)
                                      -pow(((PyP4+PyP6)),2)-pow(((PzP4+PzP6)),2)))&&
                ((pow(((EnP2+EnP5)),2)-pow(((PxP2+PxP5)),2)
                                      -pow(((PyP2+PyP5)),2)-pow(((PzP2+PzP5)),2))>
                 (pow(((EnP5+EnP6)),2)-pow(((PxP5+PxP6)),2)
                                      -pow(((PyP5+PyP6)),2)-pow(((PzP5+PzP6)),2)))&&
                ((pow(((EnP3+EnP4)),2)-pow(((PxP3+PxP4)),2)
                                      -pow(((PyP3+PyP4)),2)-pow(((PzP3+PzP4)),2))>
                 (pow(((EnP4+EnP6)),2)-pow(((PxP4+PxP6)),2)
                                      -pow(((PyP4+PyP6)),2)-pow(((PzP4+PzP6)),2)))&&
                ((pow(((EnP3+EnP5)),2)-pow(((PxP3+PxP5)),2)
                                      -pow(((PyP3+PyP5)),2)-pow(((PzP3+PzP5)),2))>
                 (pow(((EnP5+EnP6)),2)-pow(((PxP5+PxP6)),2)
                                      -pow(((PyP5+PyP6)),2)-pow(((PzP5+PzP6)),2))))"
\end{verbatim} 

The {\tt FSModeInfo} object also contains a list of ``categories'' that are used by the {\tt FSModeCollection} class~(next section) for organization.  The {\tt display} method shows information about a given mode, including a list of categories, some of which are added by default.  In the following, the first use of {\tt display} will show the default list of categories; the second will also show the added categories:
\begin{verbatim}
    > mi.display();
      // ==>   0    1_220111    K+  K+  K-  K-  pi+  pi-  pi0  eta 
      // ==>                      Hadronic  HasGammas  HasEtas  HasKaons  
      // ==>                      HasPions  HasPi0s  1Eta4K2Pi1Pi0  8Body  
      // ==>                      4Gamma  CODE=1_220111  CODE1=220111  CODE2=1
    > mi.addCategory("TEST1");
    > mi.addCategory("TEST2");
    > mi.display();
      // ==>   0    1_220111    K+  K+  K-  K-  pi+  pi-  pi0  eta 
      // ==>                      Hadronic  HasGammas  HasEtas  HasKaons  
      // ==>                      HasPions  HasPi0s  1Eta4K2Pi1Pi0  8Body  
      // ==>                      4Gamma  CODE=1_220111  CODE1=220111  CODE2=1
      // ==>                      TEST1  TEST2
\end{verbatim}

\subsection{Collections of Modes: the {\tt FSModeCollection} class}
\label{sec:modecollection}

A list of final states~({\tt FSModeInfo} objects) is managed by the {\tt FSModeCollection} class through static member functions.  The {\tt FSModeCollection} class uses the categories associated with different final states to produce sublists.  The initial list of final states is empty.  There are a few methods to add final states to the list.  Here is one, where the optional additions to the end of the final state strings add categories:
\begin{verbatim}
    > FSModeCollection::addModeInfo("K+ K-         2K  phi   EXA");
    > FSModeCollection::addModeInfo("pi+ pi-       2pi rho   EXB");
    > FSModeCollection::addModeInfo("pi+ pi- pi0   3pi omega EXA");
\end{verbatim}

The {\tt display} method will show final states associated with different combinations of categories.  Boolean operators~(and = {\tt\&} or {\tt\&\&}; or = {\tt,} or {\tt||}; not = {\tt!}) and wildcards~({\tt*} and {\tt?}) are allowed:
\begin{verbatim}
    > FSModeCollection::display();                // shows all three
    > FSModeCollection::display("2K");            // shows just the 2K mode
    > FSModeCollection::display("2K,2pi");        // shows 2K and 2pi
    > FSModeCollection::display("EXA");           // shows 2K and 3pi
    > FSModeCollection::display("EX*");           // shows all three
    > FSModeCollection::display("2K&2pi");        // shows none
    > FSModeCollection::display("2K&!2pi");       // shows 2K
    > FSModeCollection::display("HasPions");      // shows 2pi,3pi
    > FSModeCollection::display("HasPions&!3pi"); // shows 2pi
\end{verbatim}

The same list could be created from a text file (the format resembles that for {\tt EvtGen}; blank lines are ignored; everything after a {\tt \#} is ignored):
\begin{verbatim}
        ---- file: ThreeModes.modes ----
    Decay ThreeModes
      K+ K-          2K  phi   EXA
      pi+ pi-        2pi rho   EXB
      pi+ pi- pi0    3pi omega EXA
    Enddecay
        ----------- end file -----------
    > FSModeCollection::addModesFromFile("ThreeModes.modes");
\end{verbatim}
Nested lists can also be used:
\begin{verbatim}
        ---- file: NestedModes.modes ----
    Decay psi'
      pi+ pi- JPSI     pipiJpsi
      eta JPSI         etaJpsi
    Enddecay
    Decay JPSI
      mu+ mu-      MM
      e+ e-        EE
    Enddecay
        ----------- end file -----------
    > FSModeCollection::addModesFromFile("NestedModes.modes");
    > FSModeCollection::display("");  // shows all four combinations 
    > FSModeCollection::display("MM");  // shows two modes with mu+ mu- 
    > FSModeCollection::display("etaJpsi");  // shows two modes with eta JPSI 
    > FSModeCollection::display("etaJpsi&MM");  // shows one mode 
\end{verbatim}

Other methods also operate on combinations of categories:
\begin{verbatim}
    > FSModeCollection::addModesFromFile("NestedModes.modes");
    > vector<FSModeInfo*> modes = FSModeCollection::modeVector("MM");
    > TString FN("file.MODECODE.root");  TString NT("tree_MODECODE");
    > for (unsigned int i = 0; i < modes.size(); i++){ 
    >   cout << "a file name:  " << modes[i]->modeString(FN) << endl;
    >   cout << "a tree name:  " << modes[i]->modeString(NT) << endl;
    > }
\end{verbatim}


\subsection{Histograms for Multiple Modes: the {\tt FSModeHistogram} class}
\label{sec:modehist}

The {\tt FSModeHistogram} class combines features from the classes described above to make histograms for multiple final states and to manage the particle combinatorics within those final states.

The primary member function is {\tt FSModeHistogram::getTH1F}, which closely resembles the {\tt FSHistogram::getTH1F} function described in Sec.~\ref{sec:hist}.  It takes an additional argument that specifies the modes to loop over.  In addition to {\tt FSTree::expandVariable}, it also incorporates methods like {\tt ModeInfo::modeString}, {\tt ModeInfo::modeCombinatorics}, {\tt ModeInfo::modeCuts}, and {\tt ModeCollection::modeVector}, all illustrated above. 

Here is an example that would plot the mass of the $J/\psi$ given the decay modes specified by the {\tt NestedModes.modes} file shown in the previous section.  The first histogram is a sum of two histograms; the second and third histograms are a sum of four histograms.
\begin{verbatim}
    > FSModeCollection::addModesFromFile("NestedModes.modes");
    > TString FN("file.MODECODE.root");  TString NT("tree_MODECODE");
    >     // plot the total mu+ mu- mass for modes containing muons  
    > FSModeHistogram::getTH1F(FN,NT,"MM","MASS([mu+],[mu-])","(100,3.0,3.2)",
        "Chi2DOF<5.0")->Draw();
    >     // plot the total di-lepton mass for all four modes  
    > FSModeHistogram::getTH1F(FN,NT,"","MASS([l+],[l-])","(100,3.0,3.2)",
        "Chi2DOF<5.0")->Draw();
    >     // plot the same using a cut on track quality
    > TString cutTRACK("AND(TrackQualityParticle[tk]>10)");
    > FSModeHistogram::getTH1F(FN,NT,"","MASS([l+],[l-])","(100,3.0,3.2)",
        "Chi2DOF<5.0"+AND+cutTRACK)->Draw();
\end{verbatim}

To explicitly see how the histograms are constructed, use:
\begin{verbatim}
    > FSControl::DEBUG = true;
\end{verbatim}

The histogram caches also work here:  methods like {\tt FSHistogram::dumpHistogramCache} work as before.

\subsection{Information about MC Components}
\label{sec:mccomponents}

The {\tt FSModeHistogram} class also includes methods that operate on Monte Carlo truth information.  These methods assume trees include variables called {\tt MCDecayCode1}, {\tt MCDecayCode2}, and {\tt MCExtras} that contain truth information about the final state contents.  {\tt MCDecayCode1} and {\tt MCDecayCode2} have the same format as {\tt modeCode1} and {\tt modeCode2} described in Sec.~\ref{sec:modeconv}. {\tt MCExtras} has the format:
\begin{verbatim}
    MCExtras = abcd
      a = # neutrinos      c = # neutrons    
      b = # K long         d = # anti-neutrons
\end{verbatim}

The {\tt FSModeHistogram::drawMCComponents} method takes the same arguments (file name, tree name, category, etc.) as the {\tt FSModeHistogram::getTH1F} method, but draws the histogram with different colors representing different MC components.  

Other methods, like {\tt FSModeHistogram::getMCComponents}, return lists of MC components, where the components are labeled by a single string with format {\tt  MCExtras\_MCDecayCode2\_MCDecayCode1}.

\subsection{Operations on Multiple Trees: the {\tt FSModeTree} class}
\label{sec:modetree}

The {\tt FSModeTree} class contains static member functions that operate on multiple trees.

The {\tt FSModeTree::skimTree} method is the same as the {\tt FSTree::skimTree} method~(Sec.~\ref{sec:tree}), except it takes an argument to specify a combination of categories.  To skim trees for the final states listed in {\tt NestedModes.modes}, and to make track quality cuts on tracks, for example:
\begin{verbatim}
    > FSModeCollection::addModesFromFile("NestedModes.modes");
    > TString inFN("file.MODECODE.root");  TString NT("tree_MODECODE");
    > TString outFN("skim.MODECODE.root");
    > TString cutTRACK("AND(TrackQualityParticle[tk]>10)");
    > FSModeTree::skimTree(inFN,NT,"EE,MM",outFN,cutTRACK);
\end{verbatim}

It often happens in particle physics experiments that a given event can be reconstructed multiple times under different hypotheses.  This can happen within a single final state -- for example, an event from the final state $K^+K^-\pi^+\pi^-\pi^0$ could be reconstructed once correctly and again one or more times incorrectly by misidentifying pions as kaons and vice versa.  It can also happen across several final states -- for example, the same event from the $K^+K^-\pi^+\pi^-\pi^0$ final state could also be reconstructed as $K^+K^-\pi^+\pi^-$ by missing the $\pi^0$.  

The different hypotheses in the above scenarios would lead to different values of the $\chi^2$ from kinematic fits.  The {\tt createChi2RankingTree} method uses the {\tt TTree} variables {\tt Run}, {\tt Event}, and {\tt Chi2} to rank hypotheses by $\chi^2$.  It does this by creating a friend tree in a new file -- the name of the new file name is the same as the old file name except with a ``{\tt .Chi2Rank}'' appended.  The friend tree contains the variables:
\begin{verbatim}
    // Chi2RankCombinations:  number of combinations within a final state
    // Chi2Rank:  rank of this combination within a final state
    // Chi2RankCombinationsGlobal:  number of combinations in all final states
    // Chi2RankGlobal:  rank of this combination in all final states
\end{verbatim}
The friend tree is used automatically by {\tt FSModeHistogram} by setting:
\begin{verbatim}
    > FSTree::addFriendTree("Chi2Rank");
\end{verbatim}
A generalized version of the {\tt createChi2RankingTree} method is called {\tt createRankingTree} and works in the same way.  In this version, the variable ranked, its name, and the variables used to group combinations (like {\tt Run} and {\tt Event}) can be customized.


\section{Fitting Utilities:  the {\tt FSFit} directory}

The fitting utilities (contained in the directory {\tt FSFit}) work, but are still under development.  See the examples in the directory {\tt Examples/Fitting} for the general idea.

\section{Organizing Data and Data Sets: the {\tt FSData} directory}

The {\tt FSData} directory contains utilities to manipulate data and data sets.  The {\tt FSXYPoint} classes are general, while the {\tt FSEEDataSet} and {\tt FSEEXS} classes are specific to $e^+e^-$ data sets and cross sections, respectively.

Points can be read from files using the {\tt FSXYPointList::addXYPointsFromFile} method.
Data sets~(for $e^+e^-$) can be read from files using the {\tt FSEEDataSetList::addDataSetsFromFile} method.  Cross sections~(for $e^+e^-$) are read using the {\tt FSEEXSList::addXSFromFile} method.  The resulting lists can then be manipulated using ``categories'' (in a way similar to the way final states are manipulated by the {\tt FSMode} classes).  A selection of data sets and reactions from BESIII can be found in the files {\tt BESLUMINOSITIES.txt} and {\tt BESREACTIONS.txt}, respectively.

\end{document}
