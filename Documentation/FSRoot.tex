\documentclass[11pt]{article}
%\usepackage{graphicx}
\usepackage{amssymb}
%\usepackage[]{epsfig}
%\usepackage{parskip}

\textheight 8.0in
\topmargin 0.0in
\textwidth 6.5in
\oddsidemargin 0.0in
\evensidemargin 0.0in

\newcommand{\FSR}{{\tt FSRoot}}
\newcommand{\ROOT}{{\tt ROOT}}
\newcommand{\git}{{\tt git}}

\begin{document}


\title{Notes on the \FSR\ Package}
\author{Ryan Mitchell}
\date{\today}
\maketitle

\abstract{
\FSR\ is a set of utilities to help manipulate information about different Final States~(FS) produced in particle physics experiments.  The utilities are built around the CERN \ROOT\ framework.  This document provides an introduction to \FSR.}

\tableofcontents

\parindent 0pt
\parskip 10pt



%\section{Overview of \FSR}

%This is an overview of \FSR.

\section{Installation and Initial Setup}

Instructions for installation and initial setup:

(1) Download the source:
\begin{verbatim}
    > git clone https://github.com/remitche66/FSRoot.git FSRoot
\end{verbatim}

(2) Set the location of \FSR\ in your login shell script (e.g. {\tt .cshrc}):
\begin{verbatim}
    setenv FSROOT [xxxxx]/FSRoot
\end{verbatim}

(3) Also probably add the \FSR\ directory to {\tt \$DYLD\_LIBRARY\_PATH} and {\tt \$LD\_LIBRARY\_PATH}.  This allows you to compile code including \FSR\ functions.  For example:
\begin{verbatim}
     setenv DYLD_LIBRARY_PATH $DYLD_LIBRARY_PATH\:$FSROOT
     setenv   LD_LIBRARY_PATH   $LD_LIBRARY_PATH\:$FSROOT
\end{verbatim}

(4) There is usually a {\tt .rootrc} file in your home directory that \ROOT\ uses for initialization.  Add lines like these to {\tt .rootrc}, which tell \ROOT\ the location of \FSR:
\begin{verbatim}
    Unix.*.Root.DynamicPath: .:$(FSROOT):$(ROOTSYS)/lib:
    Unix.*.Root.MacroPath:   .:$(FSROOT):
\end{verbatim}

(5) Now when you open \ROOT, the \FSR\ utilities should be loaded and compiled -- you should see a message saying "Loading the FSRoot Macros" along with the output of the compilation.  The loading and compiling is done in the file {\tt rootlogon.C}.  This file also sets up default styles, which are not essential.  Since these might conflict with styles you have defined elsewhere, it could be worthwhile to tweak or remove these.

\section{Conventions}



\section{Basic Histograms}

Basic histogram functions are provided by the {\tt FSHistogram} class in the {\tt FSBasic} directory.  Like many other functions within \FSR, the functions within the {\tt FSHistogram} class are static member functions, so there is never a need to deal with instances of {\tt FSHistogram}.

The basic functionality is through the {\tt FSHistogram::getTH1F} and {\tt FSHistogram::getTH2F} classes.  Here are example uses that can be run from either the \ROOT\ command line or from a macro~(see {\tt Examples/Intro/intro.C}).  The first draws a 1d histogram and the second draws a 2d histogram.  The third argument is the variable to plot; the fourth holds the number of bins and bounds.
\begin{verbatim}
    > FSHistogram::getTH1F(FileName,TreeName,"Chi2DOF","(60,0.0,6.0)","")->Draw();
    > FSHistogram::getTH2F(FileName,TreeName,"Chi2DOF:Event",
          "(100,0.0,1.0e6,60,0.0,6.0)","")->Draw("colz");
\end{verbatim}
Cuts can be added in the fifth argument:
\begin{verbatim}
    > FSHistogram::getTH1F(FileName,TreeName,"Chi2DOF",
        "(60,0.0,6.0)","Chi2DOF<5.0")->Draw();
\end{verbatim}
The variable and cut arguments can contain shortcuts.  For example, {\tt MASS(1,2)} is expanded into the total invariant mass of particles 1 and 2, where 1 and 2 are not necessarily numbers.  This is done in the member function {\tt expandVariable}, which can also be used on its own to explicitly see what it is doing.  Characters in front of {\tt MASS}~(for example) are prepended to the variable names.  See {\tt FSHistogram::expandVariable} for more options.  Examples:
\begin{verbatim}
    > FSHistogram::getTH1F(FileName,TreeName,"MASS(1,2)",
        "(60,0.0,6.0)","Chi2DOF<5.0&&MASS(1,2)>1.0")->Draw();
    > cout << FSHistogram::expandVariable("MASS(1,2)+MASS(2,3)") << endl;
    > cout << FSHistogram::expandVariable("ABMASS(1,C)") << endl;
    > cout << FSHistogram::expandVariable("RECOILMASS(CM;D,2)") << endl;
\end{verbatim}
Histograms are automatically cached so they are made only once.  To save histograms at the end of a session, use the function:
\begin{verbatim}
    > FSHistogram::dumpHistogramCache(); 
\end{verbatim}
To read in the cache at the beginning of a session:
\begin{verbatim}
    > FSHistogram::readHistogramCache(); 
\end{verbatim}
To clear a cache from memory during a session:
\begin{verbatim}
    > FSHistogram::clearHistogramCache(); 
\end{verbatim}
To see more verbose output during a session:
\begin{verbatim}
    > FSControl::DEBUG = true;
\end{verbatim}


\end{document}
